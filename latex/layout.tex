%!TEX root = std.tex
%% layout.tex -- set overall page appearance

%%--------------------------------------------------
%%  set page size, type block size, type block position

\setlrmarginsandblock{2.245cm}{2.245cm}{*}
\setulmarginsandblock{2.5cm}{2.5cm}{*}

%%--------------------------------------------------
%%  set header and footer positions and sizes

\setheadfoot{\onelineskip}{2\onelineskip}
\setheaderspaces{*}{2\onelineskip}{*}

%%--------------------------------------------------
%%  make miscellaneous adjustments, then finish the layout
\setmarginnotes{7pt}{7pt}{0pt}
\checkandfixthelayout

%%--------------------------------------------------
%% If there is insufficient stretchable vertical space on a page,
%% TeX will not properly consider penalties for a good page break,
%% even if \raggedbottom (default) is in effect.
\addtolength{\topskip}{0pt plus 20pt}

%%--------------------------------------------------
%% Place footnotes at the bottom of the page, rather
%% than immediately following the main text.
\feetatbottom

%%--------------------------------------------------
%% Paragraph and bullet numbering

% create a new counter that resets for each new subclause
\newcommand{\newsubclausecounter}[1]{
\newcounter{#1}
\counterwithin{#1}{chapter}
\counterwithin{#1}{section}
\counterwithin{#1}{subsection}
\counterwithin{#1}{subsubsection}
\counterwithin{#1}{paragraph}
\counterwithin{#1}{subparagraph}
}

\newsubclausecounter{Paras}
\newcounter{Bullets1}[Paras]
\newcounter{Bullets2}[Bullets1]
\newcounter{Bullets3}[Bullets2]
\newcounter{Bullets4}[Bullets3]

\makeatletter
\newcommand{\parabullnum}[2]{%
\stepcounter{#1}%
\noindent\makebox[0pt][l]{\makebox[#2][r]{%
\scriptsize\raisebox{.7ex}%
{%
\ifnum \value{Paras}>0
\ifnum \value{Bullets1}>0 (\fi%
                          \arabic{Paras}%
\ifnum \value{Bullets1}>0 .\arabic{Bullets1}%
\ifnum \value{Bullets2}>0 .\arabic{Bullets2}%
\ifnum \value{Bullets3}>0 .\arabic{Bullets3}%
\fi\fi\fi%
\ifnum \value{Bullets1}>0 )\fi%
\fi%
}%
\hspace{\@totalleftmargin}\quad%
}}}
\makeatother

% Register our intent to number the next paragraph. Don't actually number it
% yet, because we might have a paragraph break before we see its contents (for
% example, if the paragraph begins with a note or example).
\def\pnum{%
\global\def\maybeaddpnum{\global\def\maybeaddpnum{}\parabullnum{Paras}{0pt}}%
\everypar=\expandafter{\the\everypar\maybeaddpnum}%
}

% Leave more room for section numbers in TOC
\cftsetindents{section}{1.5em}{3.0em}
